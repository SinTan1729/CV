\documentclass{article}
\usepackage[a4paper, margin=2cm]{geometry}
\usepackage{amsmath}
\usepackage{amsfonts}
\usepackage{dirtytalk}
\usepackage{enumitem}
\usepackage{array, xcolor}
\usepackage{longtable}
\usepackage{lastpage}
\usepackage{fancyhdr}

\fancyhf{}
\fancyhead[L]{Curriculum Vitae}\fancyhead[R]{Sayantan Santra}
\fancyfoot[R]{\footnotesize Page \thepage \space of \pageref{LastPage}}
\pagestyle{fancy}

\fancypagestyle{plain}{
  \renewcommand{\headrulewidth}{0pt}
  \fancyhf{}
  \fancyfoot[R]{\footnotesize Page \thepage\ of \pageref{LastPage}}
}

\definecolor{lightgray}{gray}{0.8}
\newcolumntype{L}{>{\raggedleft}p{0.18\textwidth}}
\newcolumntype{R}{p{0.76\textwidth}}
\newcommand\VRule{\color{lightgray}\vrule width 0.5pt}
\hyphenchar\font=-1

\title{\bfseries \Huge Curriculum Vitae}
\author{}
\date{}

\begin{document}
\maketitle
\vspace*{-2cm}
\section*{Personal Information}
\begin{tabular}{L!{\VRule}R}
	Name      & Sayantan Santra                                                               \\
	Email     & sayantan.santra689@gmail.com                                                  \\
	          & sayantan.santra@ou.edu                                                        \\
	Address   & Norman, Oklahoma, US, ZIP - 73071                                             \\
	Languages & Bangla (native), English (fluent), Hindi (fluent) and Sanskrit (intermediate)
\end{tabular}
\section*{Research Interests}
I'm interested in Algebraic and Analytic Number Theory, specifically L-functions of modular forms and elliptic curves.
\section*{Work Experience}
\begin{tabular}{L!{\VRule}R}
	2021-(ongoing) & Graduate Teaching Assistant at The University of Oklahoma, Norman, OK, US
\end{tabular} \\
\section*{Education}
\begin{tabular}{L!{\VRule}R}
	2021-(ongoing) & PhD in Mathematics from The University of Oklahoma, Norman, OK, US                                                    \\
	2019-2021      & M. Math. Mathematics (First Division with Distinction) from Indian Statistical Institute, Bengaluru, Karnataka, India \\
	2016-2019      & B.Sc. (Hons.) Mathematics (First Class) from Ramakrishna Mission Residential College, Narendrapur, Kolkata, WB, India \\
	2016           & Grade O in Higher Secondary examination                                                                               \\
	2014           & Grade AA in Secondary Examination
\end{tabular} \\
\section*{Ranks and Scholarships}
\begin{longtable}{L!{\VRule}R}
	2016 & Received the prestigious INSPIRE SHE scholarship (for the period 2016-2019)                                                                                                        \\
	2019 & Secured nationwide rank 87 in JAM 2019 and was among the 5 students selected for admission in the prestigious IISc (Indian Institute of Science) for integrated PhD in Mathematics \\
	2019 & Was among the 15 students selected for interview at the prestigious CMI (Chennai Mathematical Institute) for M.Sc. in Mathematics                                                  \\
	2019 & Secured nationwide rank 2 in the M. Math. entrance test of ISI (Indian Statistical Institute)                                                                                      \\
	2019 & Secured nationwide rank 10 in CSIR-UGC NET, December 2019.                                                                                                                         \\
	2019 & Got into the top quartile in Simon Marais Mathematics Competition and received a special mention                                                                                   \\
	2021 & Was among the 9 students selected for PhD at the prestigious TIFR (Tata Institute for Fundamental Research), Mumbai
\end{longtable}
\section*{Computer Skills}
I know the languages SageMath, Rust, C++, C, Python, R, JavaScript, HTML, \LaTeX, SQL. I'm also familiar with Linux and shell programming.
\section*{Projects and Internships}
\begin{tabular}{L!{\VRule}R}
	2015 & INSPIRE Internship during class XI                                                                                                           \\
	2017 & NPTEL course in Graph Theory (2017) : Got 93\% in the certification exam                                                                     \\
	2020 & Category Theory course under Dr. Amit Kuber of IIT Kanpur in September                                                                       \\
	2021 & Project titled \say{Primes of the form $p=x^2+ny^2$} under the guidance of Dr. Ramesh Sreekantan of ISI Bengaluru during January-May         \\
	2022 & Reading course on Modular Forms under the guidance of Dr. Kimball Martin of the University of Oklahoma during January-May                    \\
	2022 & Research project on L-functions of elliptic curves under the guidance of Dr. Kimball Martin of the University of Oklahoma during May-August.
\end{tabular}
\section*{Conferences Attended}
\begin{tabular}{L!{\VRule}R}
	2016,17,18 & Analytica at St. Xavier's College, Kolkata, India                                                                              \\
	2022       & TORA (Texas-Oklahoma Representations and Automorphic forms) XI from April 1-3 at Oklahoma State University, Stillwater, OK, US
\end{tabular}
\section*{Standardised Tests}
\begin{itemize}
	\item {\bf TOEFL iBT (November 2019)} \\
	      Reading (29), Listening (30), Speaking (25) and Writing (26) \\
	      Total - {\bf 110/120}
\end{itemize}
\section*{Extracurricular Activities}
I read and write poetry, and read literature. I have published several poems in both my native language Bangla and English. I also like to participate in Quiz. I have competed in state level competitions and won prizes in district level competitions. I love listening to music and watching movies. I think music is an universal language, much like mathematics, that helps us connect with our world.
\section*{Declaration}
I hereby declare that the details and information given above are complete and true to the best of my knowledge. \\
\vspace{2cm} \\
Date \,: June 9, 2022 \hfill [SAYANTAN SANTRA] \\
Place  : Norman, OK, US \hfill Signature \hspace{1cm} \\
\end{document}