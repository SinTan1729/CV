\documentclass{article}
\usepackage[a4paper, margin=2cm]{geometry}
\usepackage{amsmath}
\usepackage{amsfonts}
\usepackage{dirtytalk}
\usepackage{enumitem}
\usepackage{array, xcolor}
\usepackage{longtable}
\usepackage{lastpage}
\usepackage{fancyhdr}
\usepackage{hyperref, bookmark}

% Define headers and footers
\fancyhf{}
\fancyhead[L]{Curriculum Vitae}\fancyhead[R]{Sayantan Santra}
\fancyfoot[R]{\footnotesize Page \thepage \space of \pageref{LastPage}}
\pagestyle{fancy}
% Define footers for the first page
\fancypagestyle{plain}{
	\renewcommand{\headrulewidth}{0pt}
	\fancyhf{}
	\fancyfoot[R]{\footnotesize Page \thepage\ of \pageref{LastPage}}
}

\definecolor{lightgray}{gray}{0.8}
\newcolumntype{L}{>{\raggedleft}p{0.18\textwidth}}
\newcolumntype{R}{p{0.76\textwidth}}
\newcommand\VRule{\color{lightgray}\vrule width 0.5pt}

\title{\bfseries \Huge Curriculum Vitae}
\author{}
\date{}

\begin{document}
\maketitle
\vspace*{-2cm}

\section{Personal Information}
\begin{tabular}{L!{\VRule}R}
	Name      & Sayantan Santra                                                               \\
	Email     & mail@sayantansantra.com                                                       \\
	          & sayantan.santra@ou.edu                                                        \\
	Address   & Norman, Oklahoma, US, ZIP - 73071                                             \\
	Languages & Bangla (native), English (fluent), Hindi (fluent) and Sanskrit (intermediate) \\
\end{tabular}

\section{Research Interests}
I'm interested in Algebraic Number Theory, specifically L-functions of modular forms and elliptic curves.

\section{Work Experience}
\subsection{The University of Oklahoma, Norman, OK, US (ongoing since 2021)}
I currently work here as a Graduate Teaching Assistant. I've served as a discussion leader or grader in many semesters. Below are the
classes that I taught as an Instructor of Record: \\
\vspace*{1pt} \\
\begin{tabular}{L!{\VRule}R}
	Summer 2024 & Calculus and Analytic Geometry III (Math-2433) \\
	Fall 2023   & Precalculus and Trigonometry (Math-1523)       \\
	Summer 2022 & Precalculus and Trigonometry (Math-1523)       \\
\end{tabular}

\section{Education}
\begin{tabular}{L!{\VRule}R}
	2021-     & PhD in Mathematics from The University of Oklahoma, Norman, OK, US                                                    \\
	2019-2021 & M. Math. Mathematics (First Division with Distinction) from Indian Statistical Institute, Bengaluru, Karnataka, India \\
	2016-2019 & B.Sc. (Hons.) Mathematics (First Class) from Ramakrishna Mission Residential College, Narendrapur, Kolkata, WB, India \\
	2016      & Grade O in Higher Secondary examination                                                                               \\
	2014      & Grade AA in Secondary Examination                                                                                     \\
\end{tabular}

\section{Awards and Achievements}
\begin{longtable}{L!{\VRule}R}
	Spring 2024 & I was a recipient of the 2024 C. Eugene Springer Scholarship.                                                                   \\
	Spring 2021 & I was among the 9 students selected for PhD at the prestigious TIFR (Tata Institute for Fundamental Research), Mumbai.          \\
	Fall 2019   & I got into the top quartile in Simon Marais Mathematics Competition and received a special mention.                             \\
	Fall 2019   & I secured a national rank of 10 in CSIR-UGC NET, December 2019.                                                                 \\
	Spring 2019 & I secured a national rank of 2 in the M. Math. entrance test of ISI (Indian Statistical Institute).                             \\
	Spring 2019 & I was among the 15 students selected for interview at the prestigious CMI (Chennai Mathematical Institute) for M.Sc. in
	Mathematics.                                                                                                                                  \\
	Spring 2019 & I secured a national rank of 87 in JAM 2019 and was among the 5 students selected for admission in the prestigious IISc (Indian
	Institute of Science) for integrated PhD in Mathematics.                                                                                      \\
	2016        & I received the prestigious INSPIRE SHE scholarship (for the period 2016-2019).                                                  \\
\end{longtable}

\section{Projects and Internships}
\begin{tabular}{L!{\VRule}R}
	Summer 2023 & Formalization of Mathematics Summer School from 5-16 June in SLMath (formerly MSRI), Berkeley, CA                                 \\
	            & [We learned about the Lean 4 proof assistant and I, as part of a team, did a project to prove a bunch of theorems related to
	Krull dimension to be put into Mathlib.]                                                                                                        \\
	Fall 2022   & PAWS (Preliminary Arizona Winter School): Heights in Diophantine Geometry under the guidance of Dr. Padmavathi Srinivasan of
	ICERM                                                                                                                                           \\
	Summer 2022 & Research project on statistical trends of coefficients of L-functions of elliptic curves under the guidance of Dr. Kimball Martin
	of the University of Oklahoma                                                                                                                   \\
	Spring 2021 & Project titled \say{Primes of the form \(p=x^2+ny^2\)} under the guidance of Dr. Ramesh Sreekantan of ISI Bengaluru               \\
	Fall 2020   & Category Theory course under Dr. Amit Kuber of IIT Kanpur                                                                         \\
	Summer 2017 & NPTEL course in Graph Theory (2017) : Got 93\% in the certification exam                                                          \\
	2015        & INSPIRE Internship during class XI                                                                                                \\
\end{tabular}

\section{Programming}
I have experience with the languages SageMath, Lean 4, Rust, Haskell, C, Python, OCaml, JavaScript, C++, HTML, \LaTeX, and some SQL. I'm also
familiar with Linux, Docker, and shell programming. Below are some of the projects that I developed and/or maintain:
\begin{itemize}
	\item Chhoto-URL: A simple, blazingly fast, selfhosted URL shortener with no unnecessary features; written in Rust.
	\item movie-rename: A simple tool to rename movies, written in Rust.
	\item TvTimeToTrakt: A Python script to import TV Time data into Trakt.TV
	\item recipe-box-for-wikijs: Download recipes into a (repurposed Zettelkasten) recipe box.
\end{itemize}

\section{Presentations/Talks}
\begin{tabular}{L!{\VRule}R}
	Fall 2024   & \say{Mod \(\ell\) Modular Forms} in Student Number Theory Seminar at the University of Oklahoma                                   \\
	Fall 2024   & \say{Cubic and Biquadratic Reciprocity} in Number Theory Learning Seminar at the University of Oklahoma                           \\
	Spring 2024 & \say{Proof Formalization in Lean, or: how to trick your computer into doing (even) more math} together with Dr. Mario Morán Cañón
	in OU Math Club at the University of Oklahoma                                                                                                   \\
	Spring 2024 & \say{Exceptional Primes and Where to Find Them} in MathFest at the University of Oklahoma                                         \\
	Spring 2024 & \say {A (very) Brief Introduction to Lean 4} in Student Algebra Seminar at the University of Oklahoma                             \\
	Fall 2023   & \say{On Congruences of Coefficients of Modular Forms} in the ARTS (Algebra and Representation Theory Seminar) at the University
	of Oklahoma                                                                                                                                     \\
	Fall 2023   & \say{On Congruences of Coefficients of Modular Forms} in Student Presentation Seminar at the University of Oklahoma               \\
	Spring 2023 & \say{Elliptic Curves and Integer Factorization} in Student Presentation Seminar at the University of Oklahoma                     \\
	Fall 2022   & \say{Motivations and Consequences of the Prime Number Theorem} in SPS (Student Presentation Seminar) at the University of
	Oklahoma                                                                                                                                        \\
	Fall 2021   & \say{Primes of the Form \(p=x^2+ny^2\)} in Student Algebra Seminar at the University of Oklahoma                                  \\
	Spring 2021 & \say{Primes of the Form \(p=x^2+ny^2\)} for final semester project presentation at the Indian Statistical Institute               \\
	Spring 2019 & \say{Planar Graphs and n-Holed Tori} at RKMRC Narendrapur                                                                         \\
\end{tabular}

\section{Conferences Attended}
\begin{tabular}{L!{\VRule}R}
	Fall 2024   & KS Math Graduate Student Conference at University of Kansas, Lawrence, KS, US                                            \\
	Fall 2024   & Building Bridges: 6th EU/US Summer School \& Workshop on Automorphic Forms and Related Topics at Centre International de
	Rencontres Mathématiques (CIRM), Marseille, France                                                                                     \\
	Spring 2024 & TORA (Texas-Oklahoma Representations and Automorphic forms) XIII at University of North Texas, Denton, TX, US            \\
	Spring 2024 & Hybrid Conference on AI-Math organized by UERJ, RJ, Brazil                                                               \\
	Fall 2023   & TORA (Texas-Oklahoma Representations and Automorphic forms) XII at University of Oklahoma, Norman, OK, US                \\
	Spring 2023 & SLAM (Southwest Local Algebra Meeting) 2023 at University of North Texas, Denton, TX, US                                 \\
	Fall 2022   & TORA (Texas-Oklahoma Representations and Automorphic forms) XI at Oklahoma State University, Stillwater, OK, US          \\
	2016,17,18  & Analytica at St. Xavier's College, Kolkata, India                                                                        \\
\end{tabular}

\section{Standardised Tests}
\begin{itemize}
	\item {\bf TOEFL iBT (November 2019)} \\
	      Reading (29), Listening (30), Speaking (25) and Writing (26) \\
	      Total - {\bf 110/120}
\end{itemize}

\section{Organization and Leadership}
\begin{tabular}{L!{\VRule}R}
	Fall 2024   & Represented the Department of Mathematics as a Senator in the Graduate Student Senate of the University of Oklahoma           \\
	Fall 2024   & Organized the Student Algebra Seminar                                                                                         \\
	Spring 2024 & Organized the Student Algebra Seminar                                                                                         \\
	Fall 2020   & Organized Mathletics (an online math event during the pandemic) with some friends from various universities                   \\
	Spring 2018 & Organized Infinity (a two day inter-college math competition organized by the Department of Mathematics and the Department of
	Statistics) as a representative of the Department of Mathematics                                                                            \\
\end{tabular}

\section{Extracurricular Activities}
\begin{enumerate}[label=(\alph*)]
	\item I'm a hobbyist programmer and a firm supporter of the FOSS philosophy. All code written by me is released under GPLv3 license
	      unless compelled by some prior agreement.
	\item I'm a coffee enthusiast. I love making and drinking coffee, especially espresso. I also have interest in learning about the
	      science and history of coffee.
	\item I like to participate in quizzes, having competed in state level competitions while in high school in India.
	\item I'm also interested in literature. I've published several poems in both my native language Bangla and English.
\end{enumerate}

\section{Declaration}
I hereby declare that the details and information given above are complete and true to the best of my knowledge.
\vspace*{2cm} \\
Date \,: \today \hfill [SAYANTAN SANTRA] \\
Place  : Norman, OK, US \hfill Signature \hspace{1cm}
\end{document}
