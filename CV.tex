\documentclass{article}
\usepackage[a4paper, margin=2cm]{geometry}
\usepackage{amsmath}
\usepackage{amsfonts}
\usepackage{dirtytalk}
\usepackage{enumitem}
\usepackage{array, xcolor}
\usepackage{longtable}
\usepackage{lastpage}
\usepackage{fancyhdr}

\fancyhf{}
\fancyhead[L]{Curriculum Vitae}\fancyhead[R]{Sayantan Santra}
\fancyfoot[R]{\footnotesize Page \thepage \space of \pageref{LastPage}}
\pagestyle{fancy}

\fancypagestyle{plain}{
  \renewcommand{\headrulewidth}{0pt}
  \fancyhf{}
  \fancyfoot[R]{\footnotesize Page \thepage\ of \pageref{LastPage}}
}

\definecolor{lightgray}{gray}{0.8}
\newcolumntype{L}{>{\raggedleft}p{0.18\textwidth}}
\newcolumntype{R}{p{0.76\textwidth}}
\newcommand\VRule{\color{lightgray}\vrule width 0.5pt}
\hyphenchar\font=-1

\title{\bfseries \Huge Curriculum Vitae}
\author{}
\date{}

\begin{document}
\maketitle
\vspace*{-2cm}

\section*{Personal Information}
\begin{tabular}{L!{\VRule}R}
	Name      & Sayantan Santra                                                               \\
	Email     & sayantan.santra689@gmail.com                                                  \\
	          & sayantan.santra@ou.edu                                                        \\
	Address   & Norman, Oklahoma, US, ZIP - 73071                                             \\
	Languages & Bangla (native), English (fluent), Hindi (fluent) and Sanskrit (intermediate)
\end{tabular}

\section*{Research Interests}
I'm interested in Algebraic Number Theory, specifically L-functions of modular forms and elliptic curves.

\section*{Work Experience}
\begin{tabular}{L!{\VRule}R}
	2021- & Graduate Teaching Assistant at The University of Oklahoma, Norman, OK, US
\end{tabular} \\

\section*{Education}
\begin{tabular}{L!{\VRule}R}
	2021-     & PhD in Mathematics from The University of Oklahoma, Norman, OK, US                                                    \\
	2019-2021 & M. Math. Mathematics (First Division with Distinction) from Indian Statistical Institute, Bengaluru, Karnataka, India \\
	2016-2019 & B.Sc. (Hons.) Mathematics (First Class) from Ramakrishna Mission Residential College, Narendrapur, Kolkata, WB, India \\
	2016      & Grade O in Higher Secondary examination                                                                               \\
	2014      & Grade AA in Secondary Examination
\end{tabular} \\

\section*{Achievements}
\begin{longtable}{L!{\VRule}R}
	2021 & I was among the 9 students selected for PhD at the prestigious TIFR (Tata Institute for Fundamental Research), Mumbai.                                                                \\
	2019 & I got into the top quartile in Simon Marais Mathematics Competition and received a special mention.                                                                                   \\
	2019 & I secured a nationwide rank 10 in CSIR-UGC NET, December 2019.                                                                                                                        \\
	2019 & I secured nationwide rank 2 in the M. Math. entrance test of ISI (Indian Statistical Institute).                                                                                      \\
	2019 & I was among the 15 students selected for interview at the prestigious CMI (Chennai Mathematical Institute) for M.Sc. in Mathematics.                                                  \\
	2019 & I secured nationwide rank 87 in JAM 2019 and was among the 5 students selected for admission in the prestigious IISc (Indian Institute of Science) for integrated PhD in Mathematics. \\
	2016 & I received the prestigious INSPIRE SHE scholarship (for the period 2016-2019).
\end{longtable}

\section*{Technical Skills}
I know the languages SageMath, Lean 4, Rust, C++, C, Python, JavaScript, HTML, \LaTeX, and some SQL. I'm also familiar with Linux and shell programming.
\section*{Projects, Readings and Internships}
\begin{tabular}{L!{\VRule}R}
	2023 & Formalization of Mathematics Summer School from 5-16 June in SLMath (formerly MSRI), Berkeley, CA                                                                                          \\
	     & [We learned about the Lean 4 proof assistant and I, as part of a team, did a project to prove a bunch of theorems related to Krull dimension to be put into Mathlib.]                      \\
	2023 & Reading course on Modular Forms and Hecke Operators (following the lecture notes by Ribet-Stein) under the guidance of Dr. Kimball Martin of the University of Oklahoma during January-May \\
	2022 & Reading course on Analytical Number Theory under the guidance of Dr. Ameya Pitale of the University of Oklahoma during August-December                                                     \\
	2022 & PAWS (Preliminary Arizona Winter School): Heights in Diophantine Geometry under the guidance of Dr. Padmavathi Srinivasan of ICERM during October-November                                 \\
	2022 & Research project on statistical trends of coefficients of L-functions of elliptic curves under the guidance of Dr. Kimball Martin of the University of Oklahoma during May-August          \\
	2022 & Reading course on Modular Forms under the guidance of Dr. Kimball Martin of the University of Oklahoma during January-May                                                                  \\
	2021 & Project titled \say{Primes of the form $p=x^2+ny^2$} under the guidance of Dr. Ramesh Sreekantan of ISI Bengaluru during January-May                                                       \\
	2020 & Category Theory course under Dr. Amit Kuber of IIT Kanpur in September                                                                                                                     \\
	2017 & NPTEL course in Graph Theory (2017) : Got 93\% in the certification exam                                                                                                                   \\
	2015 & INSPIRE Internship during class XI
\end{tabular}

\section*{Presentations}
\begin{tabular}{L!{\VRule}R}
	2023 & \say{Elliptic Curves and Integer Factorization} in Student Presentation Seminar at the University of Oklahoma on 13th April                   \\
	2022 & \say{Motivations and Consequences of the Prime Number Theorem} in Student Presentation Seminar at the University of Oklahoma on 15th November \\
	2021 & \say{Primes of the Form $p=x^2+ny^2$} in Student Algebra Seminar at the University of Oklahoma in November                                    \\
	2021 & \say{Primes of the Form $p=x^2+ny^2$} for final semester project presentation at the Indian Statistical Institute in May                      \\
	2019 & \say{Planar Graphs and n-Holed Tori} at RKMRC Narendrapur in February
\end{tabular}

\section*{Conferences Attended}
\begin{tabular}{L!{\VRule}R}
	2023       & SLAM (Southwest Local Algebra Meeting) 2023 on March 4-5 at University of North Texas, Denton, TX, US                        \\
	2022       & TORA (Texas-Oklahoma Representations and Automorphic forms) XI on April 1-3 at Oklahoma State University, Stillwater, OK, US \\
	2016,17,18 & Analytica at St. Xavier's College, Kolkata, India
\end{tabular}

\section*{Standardised Tests}
\begin{itemize}
	\item {\bf TOEFL iBT (November 2019)} \\
	      Reading (29), Listening (30), Speaking (25) and Writing (26) \\
	      Total - {\bf 110/120}
\end{itemize}

\section*{Extracurricular Activities}
I'm a hobbyist programmer and an active supporter of the FOSS movement. I'm also interested in literature. I've published several poems in both my native language Bangla and English. I like to participate in Quizzes, having competed in state level competitions and won awards in district level competitions. I've helped organize many events during my college days. Also, like most people, I love music and movies.

\section*{Declaration}
I hereby declare that the details and information given above are complete and true to the best of my knowledge. \\
\vspace{2cm} \\
Date \,: \today \hfill [SAYANTAN SANTRA] \\
Place  : Norman, OK, US \hfill Signature \hspace{1cm} \\
\end{document}